\documentclass[12pt, a4paper]{article}
\usepackage[utf8]{inputenc}
\usepackage[spanish]{babel}
\usepackage{csquotes}
\usepackage[style=apa,backend=biber]{biblatex}
\addbibresource{./referencie.bib}
\usepackage[T1]{fontenc}
\usepackage{amsmath}
\usepackage{amsfonts}

\usepackage{amssymb}
\usepackage[a4paper, margin=2.54cm]{geometry}
\usepackage{setspace}
\usepackage{titlesec}
\usepackage{graphicx}
\usepackage{pgfplots}
\usepackage{pgf-pie}
\usepackage{booktabs}
% Configuración de APA 7ma edición

\setlength{\parindent}{0.5in}
\titleformat*{\section}{\normalfont\bfseries}
\titleformat*{\subsection}{\normalfont\bfseries}
\titleformat*{\subsubsection}{\normalfont\bfseries}
\pgfplotsset{compat=1.18}


\begin{document}
\onehalfspacing

\begin{titlepage}
    \begin{center}
        \vspace{0.5in}
        \small “Año del Bicentenario, de la consolidación de nuestra Independencia, y de la conmemoración de las heroicas batallas de Junín y Ayacucho”
  
        
        \LARGE Universidad Nacional Hermilio Valdizan
        
        \Large Facultad de Economía
        
        \vspace{0.5cm}
        
        \begin{figure}[h!]
            \begin{center}
                \includegraphics[width=0.40\textwidth]{images/unheval.jpg}
                \hspace{0.5cm}  
                \includegraphics[width=0.45\textwidth]{images/economia.png}
            \end{center}
        \end{figure}
        
        \LARGE{Globalización y su impacto en la economía peruana}
        \vspace{0.5cm}

        \large\textbf{Docente:}

        \large\textbf\author{Cueva Laguna, Jeel E. }

        \large\textbf{Autores:}

        \large\textbf\author{Chuquiyauri Tordecillo, Ronaldhino M.}

        \large\textbf\author{Palomino Ricaldi, Antony R.}        
        
        \vspace{0.5cm}
        
        \Large Huánuco - Perú 
        
        \large 2024
    \end{center}
\end{titlepage}

\tableofcontents
\newpage

\section{Introducción}

\subsection{Planteamiento del tema}
La globalización representa uno de los fenómenos más significativos que ha transformado la estructura económica mundial en las últimas décadas \parencite{stiglitz2002}. En el contexto peruano, este proceso ha generado profundas transformaciones en la estructura económica y social del país desde la década de 1990 \parencite{dancourt2016}. Como señalan \cite{contreras2018} y \cite{parodi2015}, la integración del Perú a la economía global ha sido un proceso complejo que ha reconfigurado no solo las relaciones comerciales y financieras, sino también el tejido social y productivo del país.

\subsection{Objetivos de la investigación}
El objetivo principal de esta investigación es analizar el impacto multidimensional de la globalización en la economía peruana, con especial énfasis en el período 1990-2024. Los objetivos específicos incluyen:

\begin{itemize}
    \item Examinar las transformaciones estructurales en la economía peruana como resultado del proceso de globalización \parencite{jimenez2017}.
    \item Evaluar los impactos en las dimensiones comercial, financiera, productiva y socioeconómica \parencite{yamada2019}.
    \item Identificar las oportunidades y desafíos que presenta la globalización para el desarrollo económico del Perú \parencite{garcia2019}.
\end{itemize}

\subsection{Justificación del estudio}
La relevancia de este estudio se fundamenta en la necesidad de comprender las implicaciones de la globalización en la economía peruana, especialmente en un contexto de cambios acelerados y crisis globales. Como argumentan \cite{mendoza2020} y \cite{castillo2018}, el análisis de la experiencia peruana en el proceso de globalización proporciona lecciones valiosas para la formulación de políticas públicas y estrategias de desarrollo. Además, \cite{seminario2019} sostiene que la comprensión de estos impactos es crucial para diseñar políticas que maximicen los beneficios y minimicen los riesgos asociados a la integración global.

\subsection{Metodología empleada}
La presente investigación adopta un enfoque metodológico cualitativo, de diseño no experimental y nivel explicativo. Se emplea una revisión sistemática de la literatura especializada, análisis de datos estadísticos de fuentes oficiales y estudios de caso específicos. Como señalan \cite{barrantes2021} y \cite{lopez2020}, este enfoque permite una comprensión más completa y matizada del fenómeno estudiado. La metodología incluye:

\begin{itemize}
    \item Análisis documental de fuentes secundarias
    \item Revisión de indicadores económicos y sociales
    \item Evaluación de políticas públicas y sus resultados
    \item Análisis comparativo con experiencias internacionales
\end{itemize}

\section{Marco Teórico}

\subsection{Conceptualización de la globalización}

\subsubsection{Definiciones principales}
La globalización constituye un fenómeno multidimensional que ha transformado las relaciones económicas, sociales y culturales a nivel mundial. \cite{held2000} la define como un proceso histórico que altera fundamentalmente los patrones de interacción social y económica mediante la creación de redes transnacionales. Por su parte, \cite{robertson2003} enfatiza que la globalización representa la intensificación de la conciencia del mundo como un todo, manifestándose en la interconexión de mercados y sociedades.

En el contexto económico, \cite{krugman2018} y \cite{sassen2007} coinciden en definir la globalización como un proceso de integración económica caracterizado por el incremento del comercio internacional, los flujos de capital y la movilidad de factores productivos. Esta perspectiva es complementada por \cite{castells2010}, quien subraya la importancia de las tecnologías de información en la configuración de una economía global interconectada.

\subsubsection{Características fundamentales}
Las características esenciales de la globalización pueden identificarse en múltiples dimensiones. \cite{dicken2015} señala como elementos fundamentales:
\begin{itemize}
    \item La compresión espacio-temporal de las actividades económicas
    \item La intensificación de los flujos transnacionales
    \item La emergencia de redes globales de producción
    \item La interdependencia de los mercados financieros
\end{itemize}

\cite{scholte2005} y \cite{harvey2009} destacan además la desterritorialización de las actividades económicas y la reconfiguración de las relaciones de poder entre estados y mercados como características definitorias del proceso globalizador.

\subsubsection{Dimensiones de la globalización}
El análisis de la globalización requiere una comprensión de sus múltiples dimensiones. \cite{keohane2018} y \cite{nye2020} identifican cuatro dimensiones principales:
\begin{itemize}
    \item Económica: Integración de mercados y sistemas productivos
    \item Política: Transformación del rol del Estado y nuevas formas de gobernanza
    \item Social: Cambios en patrones culturales y movilidad humana
    \item Tecnológica: Revolución digital y conectividad global
\end{itemize}

Estas dimensiones, según \cite{giddens2013}, no operan de manera aislada sino que se retroalimentan mutuamente, generando dinámicas complejas de transformación social y económica.

\subsection{Contexto histórico de la globalización en Perú}

\subsubsection{Antecedentes históricos}
La inserción del Perú en la economía global tiene profundas raíces históricas. \cite{thorp2012} y \cite{bulmer1998} identifican distintas etapas en la integración económica del país, desde el período colonial hasta la República. Durante el siglo XX, \cite{sheahan2001} destaca tres períodos críticos: la era del guano, el período de industrialización por sustitución de importaciones y la apertura económica de los años noventa.

\subsubsection{Proceso de apertura económica}
La apertura económica peruana representa un punto de inflexión en la historia económica del país. \cite{wise2003} y \cite{pasco2009} analizan cómo la crisis de los años ochenta precipitó un cambio radical en el modelo económico. Este proceso, según \cite{gonzales2015}, se caracterizó por:
\begin{itemize}
    \item Liberalización comercial y financiera
    \item Privatización de empresas estatales
    \item Desregulación de mercados
    \item Apertura a la inversión extranjera
\end{itemize}

\subsubsection{Reformas estructurales}
Las reformas estructurales implementadas en la década de 1990 transformaron fundamentalmente la economía peruana. \cite{abusada2000} y \cite{parodi2014} documentan las principales reformas:
\begin{itemize}
    \item Reforma comercial y arancelaria
    \item Reforma del sistema financiero
    \item Reforma laboral
    \item Reforma del Estado
\end{itemize}

\cite{franco2018} y \cite{vega2019} evalúan el impacto de estas reformas, señalando tanto sus logros en términos de estabilización macroeconómica como sus limitaciones en aspectos de equidad y desarrollo social.

\section{Dimensiones del Impacto de la Globalización}

\subsection{Dimensión Comercial}

La dimensión comercial de la globalización en el Perú ha experimentado una transformación significativa desde la década de 1990. \cite{santa_cruz2021} y \cite{rodriguez2019} señalan que esta transformación se caracteriza por una mayor apertura comercial, diversificación de mercados y modernización de la política comercial.

\subsubsection{Evolución del comercio exterior}
El comercio exterior peruano ha mostrado un crecimiento sustancial en las últimas tres décadas. Según \cite{tello2018}, el valor total del comercio exterior pasó de US\$ 6,700 millones en 1990 a más de US\$ 98,000 millones en 2019, representando un incremento en la ratio de apertura comercial (exportaciones más importaciones sobre PIB) del 22\% al 48\%. Este crecimiento, como señalan \cite{mendoza2017} y \cite{torres2020}, se ha caracterizado por:

\begin{itemize}
    \item Un aumento significativo en el volumen de exportaciones tradicionales y no tradicionales
    \item Una diversificación gradual de la canasta exportadora
    \item Un incremento en la participación de las exportaciones no tradicionales
    \item Una mayor integración en las cadenas globales de valor
\end{itemize}

La estructura del comercio exterior ha experimentado cambios significativos. \cite{leon2019} destaca que mientras en 1990 las exportaciones se concentraban en productos tradicionales (70\%), para 2020 las exportaciones no tradicionales habían aumentado su participación al 30\% del total, evidenciando una diversificación progresiva de la oferta exportable.

\subsubsection{Tratados de libre comercio}
La política comercial peruana ha priorizado la negociación y suscripción de acuerdos comerciales como estrategia de inserción internacional. \cite{ferrero2018} y \cite{novak2019} documentan que el Perú ha suscrito más de 20 acuerdos comerciales, entre los que destacan:

\begin{itemize}
    \item El Acuerdo de Promoción Comercial con Estados Unidos (2009)
    \item El Tratado de Libre Comercio con la Unión Europea (2013)
    \item El Acuerdo de Asociación Transpacífico (CPTPP)
    \item Acuerdos bilaterales con China, Japón, Corea del Sur y otros socios asiáticos
    \item Acuerdos regionales en el marco de la Comunidad Andina y la Alianza del Pacífico
\end{itemize}

\cite{garcia_belaunde2020} y \cite{morales2021} analizan el impacto de estos acuerdos, señalando que han contribuido a:
\begin{itemize}
    \item Ampliar el acceso a mercados internacionales
    \item Reducir costos de importación de bienes de capital e insumos
    \item Modernizar los marcos regulatorios
    \item Atraer inversión extranjera directa
\end{itemize}

\subsubsection{Diversificación de exportaciones}
La diversificación de las exportaciones representa uno de los desafíos más importantes para la economía peruana. \cite{vasquez2018} y \cite{alarco2021} identifican patrones significativos en este proceso:

\begin{itemize}
    \item Emergencia de nuevos sectores exportadores no tradicionales
    \item Desarrollo de cadenas de valor en agroindustria y textiles
    \item Incremento en el valor agregado de las exportaciones
    \item Mayor participación en mercados internacionales especializados
\end{itemize}

Sin embargo, \cite{fairlie2020} señala que persisten desafíos importantes:
\begin{itemize}
    \item Alta concentración en productos primarios
    \item Dependencia de mercados tradicionales
    \item Limitada incorporación de tecnología en procesos productivos
    \item Baja participación en eslabones de alto valor agregado
\end{itemize}

El análisis sectorial realizado por \cite{lopez2019} y \cite{sanchez2020} revela que los sectores más dinámicos en términos de diversificación han sido:
\begin{itemize}
    \item Agroexportación no tradicional
    \item Productos pesqueros de valor agregado
    \item Textiles y confecciones
    \item Productos químicos y metalmecánicos
    \item Servicios no financieros
\end{itemize}

\subsection{Dimensión Productiva}

La dimensión productiva de la globalización en el Perú ha generado transformaciones significativas en la estructura económica del país. \cite{tavara2020} y \cite{cespedes2019} señalan que estos cambios han afectado tanto la composición sectorial de la producción como los niveles de productividad y competitividad empresarial.

\subsubsection{Transformación de la estructura productiva}
La estructura productiva peruana ha experimentado cambios sustanciales desde la apertura económica. \cite{jimenez2021} y \cite{vera2018} identifican las siguientes transformaciones principales:

\begin{itemize}
    \item Reducción de la participación del sector industrial tradicional
    \item Expansión de los servicios modernos y el comercio
    \item Surgimiento de nuevos clusters productivos
    \item Modernización de sectores tradicionales
\end{itemize}

Sin embargo, \cite{gonzales2019} señala que esta transformación ha sido heterogénea y presenta las siguientes características:

\begin{enumerate}
    \item Concentración geográfica de la actividad productiva
    \begin{itemize}
        \item 52\% del PIB se genera en Lima y Callao
        \item Persistencia de brechas regionales en productividad
        \item Limitada articulación entre regiones
    \end{itemize}
    
    \item Dualidad estructural
    \begin{itemize}
        \item Coexistencia de sectores modernos y tradicionales
        \item Amplia brecha de productividad entre sectores
        \item Limitada transferencia tecnológica entre segmentos
    \end{itemize}
\end{enumerate}

\subsubsection{Competitividad empresarial}
La globalización ha impuesto nuevos desafíos en términos de competitividad empresarial. \cite{torres2021} y \cite{ramirez2020} analizan los principales aspectos:

\begin{itemize}
    \item Mejoras en la gestión empresarial
    \begin{itemize}
        \item Adopción de estándares internacionales
        \item Implementación de sistemas de calidad
        \item Modernización de procesos administrativos
    \end{itemize}
    
    \item Desarrollo de cadenas de valor
    \begin{itemize}
        \item Integración con proveedores globales
        \item Especialización productiva
        \item Certificaciones internacionales
    \end{itemize}
\end{itemize}

No obstante, \cite{villarreal2019} identifica importantes limitaciones:
\begin{itemize}
    \item Alta informalidad empresarial (70\% de empresas)
    \item Baja productividad laboral
    \item Limitado acceso a financiamiento
    \item Escasa inversión en investigación y desarrollo
\end{itemize}

\subsubsection{Innovación y tecnología}
El proceso de globalización ha evidenciado la importancia crítica de la innovación y la tecnología. \cite{kuramoto2018} y \cite{sagasti2020} analizan la situación del Perú en este ámbito:

\begin{itemize}
    \item Inversión en I+D
    \begin{itemize}
        \item 0.12\% del PIB (2020)
        \item Principalmente financiada por el sector público
        \item Concentrada en investigación básica
    \end{itemize}
    
    \item Adopción tecnológica
    \begin{itemize}
        \item Predominio de tecnologías importadas
        \item Limitada capacidad de adaptación
        \item Baja generación de patentes
    \end{itemize}
\end{itemize}

\cite{peters2021} y \cite{martinez2020} identifican las principales barreras para la innovación:
\begin{itemize}
    \item Limitada vinculación academia-empresa
    \item Escasez de capital humano especializado
    \item Débil infraestructura tecnológica
    \item Marco institucional insuficiente
\end{itemize}

Sin embargo, \cite{loayza2019} destaca algunas experiencias exitosas:
\begin{itemize}
    \item Desarrollo de startups tecnológicas
    \item Innovaciones en el sector agroindustrial
    \item Modernización de servicios financieros
    \item Adopción de tecnologías digitales
\end{itemize}

\end{document}