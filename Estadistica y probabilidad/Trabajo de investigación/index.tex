\documentclass[12pt, a4paper]{article}
\usepackage[utf8]{inputenc}
\usepackage[spanish]{babel}
\usepackage{csquotes}
\usepackage[style=apa,backend=biber]{biblatex}
\addbibresource{./referencie.bib}
\usepackage[T1]{fontenc}
\usepackage{amsmath}
\usepackage{amsfonts}

\usepackage{amssymb}
\usepackage[a4paper, margin=2.54cm]{geometry}
\usepackage{setspace}
\usepackage{titlesec}
\usepackage{graphicx}
\usepackage{pgfplots}
\usepackage{pgf-pie}
\usepackage{booktabs}
% Configuración de APA 7ma edición

\setlength{\parindent}{0.5in}
\titleformat*{\section}{\normalfont\bfseries}
\titleformat*{\subsection}{\normalfont\bfseries}
\titleformat*{\subsubsection}{\normalfont\bfseries}
\pgfplotsset{compat=1.18}


\begin{document}
\onehalfspacing

\begin{titlepage}
    \begin{center}
        \vspace{0.5in}
        \small “Año del Bicentenario, de la consolidación de nuestra Independencia, y de la conmemoración de las heroicas batallas de Junín y Ayacucho”
  
        
        \Large Universidad Nacional Hermilio Valdizan
        
        \Large Facultad de Economía
        
        \vspace{0.5cm}
        
        \begin{figure}[h!]
            \begin{center}
                \includegraphics[width=0.40\textwidth]{images/unheval.jpg}
                \hspace{0.5cm}  
                \includegraphics[width=0.45\textwidth]{images/economia.png}
            \end{center}
        \end{figure}
        
        \Large{Brecha digital: Un análisis descriptivo en los estudiantes de la facultad de economia de la Unheval 2024} 
        \vspace{0.5cm}

        \large\textbf{Docente:}

        \large\textbf\author{Cueva Laguna, Jeel E. }
        

        \large\textbf{Autores:}

        \large\textbf\author{Chavez Huaranga, Jorge}

        \large\textbf\author{Chuquiyauri Tordecillo, Ronaldhino M.}

        \large\textbf\author{Nieto Serpa, Luis}

        \large\textbf\author{Palomino Ricaldi, Antony R.}        
        
        \large\textbf\author{Rojas Benancio, Delsy}

        \large\textbf\author{Valentín Baquerizo, Dan Jhu Ali}
        
        \large\textbf\author{Villanueva Herrera, Tony}
        
        \vspace{0.5cm}
        
        \Large Huánuco - Perú 
        
        \large 2024
    \end{center}
\end{titlepage}


\tableofcontents
\newpage

\section{Introducción}
En la actualidad, la tecnología desempeña un papel crucial en el desarrollo educativo, profesional y social. Sin embargo, no todos los estudiantes tienen acceso equitativo a las Tecnologías de la Información y Comunicación (TIC), lo que da lugar a un fenómeno conocido como brecha digital. Este término hace referencia a las desigualdades existentes en el acceso, uso y habilidades relacionadas con las TIC, y representa un desafío significativo para las instituciones educativas que buscan formar profesionales competentes en un mundo cada vez más digitalizado.

En el caso de la Facultad de Economía de la Universidad Nacional Hermilio Valdizán (Unheval), la brecha digital afecta directamente a los estudiantes, limitando su capacidad para aprovechar las oportunidades tecnológicas disponibles en su formación académica. Estas limitaciones no solo impactan en el rendimiento académico, sino también en su preparación para enfrentar las demandas del mercado laboral, donde las competencias digitales son cada vez más valoradas. Identificar y comprender la magnitud de estas desigualdades es esencial para diseñar estrategias que permitan reducir la brecha digital y garantizar la inclusión digital de todos los estudiantes.

La presente investigación se centra en analizar la brecha digital en los estudiantes de la Facultad de Economía de la Unheval durante el ciclo académico 2024-2. Este análisis no solo busca cuantificar el acceso a dispositivos y conectividad, sino también evaluar las habilidades digitales y el uso efectivo de las TIC en el proceso de aprendizaje. A través de un enfoque multidimensional, se pretende identificar los factores que contribuyen a las desigualdades digitales y proponer soluciones que promuevan la equidad y el desarrollo académico.

La relevancia de esta investigación se sustenta en tres dimensiones principales. En primer lugar, su valor teórico radica en la contribución al conocimiento sobre la brecha digital en un contexto específico, poco explorado en investigaciones previas: los estudiantes de Economía de la Unheval. Este aporte permitirá ampliar la comprensión sobre cómo las desigualdades digitales se manifiestan en el ámbito universitario y qué implicaciones tienen para la educación superior.

En segundo lugar, la investigación tiene una implicación práctica significativa. Los resultados del estudio proporcionaron datos valiosos para identificar las áreas donde las desigualdades digitales son más pronunciadas. Con esta información, será posible diseñar políticas e intervenciones que promuevan la inclusión digital, ofreciendo igualdad de oportunidades para todos los estudiantes de la facultad. Por ejemplo, se podrían desarrollar programas de capacitación en habilidades digitales, mejorar la infraestructura tecnológica de la universidad o establecer alianzas con el sector privado para facilitar el acceso a dispositivos y conectividad.

Además, el estudio tiene un valor metodológico importante, ya que incluye el desarrollo de un instrumento de medición del acceso, uso y habilidades digitales adaptado a la realidad de los estudiantes de la Unheval. Este instrumento no solo permitirá recoger datos precisos y relevantes para el presente estudio, sino que también podrá ser utilizado en investigaciones futuras para evaluar la evolución de la brecha digital en la institución o en otros contextos similares.

El objetivo general de la investigación es analizar la brecha digital en los Estudiantes de la Facultad de Economía de la Universidad Nacional Hermilio Valdizan considerando las dimensiones de acceso, uso y habilidades digitales. Dentro de ellos encontramos los objetivos específicos: Describir el nivel de acceso a las TIC por parte de los estudiantes (dispositivos y conexión). Identificar la frecuencia y propósito del uso de las TIC. Evaluar las habilidades digitales en el uso de buscadores académicos, software de uso de hojas de cálculo, análisis de datos con software especializado y programación.

En síntesis, la brecha digital representa un obstáculo significativo para el desarrollo académico y profesional de los estudiantes de la Facultad de Economía de la Unheval. Esta investigación busca no solo diagnosticar esta problemática, sino también sentar las bases para soluciones prácticas y sostenibles que promuevan la inclusión digital y garanticen que todos los estudiantes tengan acceso a las herramientas necesarias para alcanzar su máximo potencial en la era digital.

\section{Marco Teórico}
\subsection{Antecedentes}

\textbf{Internacionales (Latinoamérica):} En América Latina, la brecha digital ha sido identificada como una de las principales barreras para el desarrollo educativo y social. Según la \textcite{cepal2020}, las desigualdades en el acceso a las TIC son evidentes entre áreas urbanas y rurales, afectando de manera desproporcionada a estudiantes de bajos ingresos. Estudios realizados por \textcite{sanchez2021} en México y Colombia evidenciaron que el acceso desigual a dispositivos tecnológicos y conectividad limita las oportunidades educativas y el desarrollo de habilidades digitales esenciales para el mercado laboral. En este contexto, los programas de inclusión digital en países como Uruguay y Chile han mostrado avances significativos, aunque persisten desafíos relacionados con la formación docente y el acceso universal.


\textbf{Nacionales (Perú):} En Perú, la brecha digital es un problema estructural que afecta particularmente a las regiones más alejadas de la capital. Según el Instituto Nacional de Estadística e Informática \parencite{inei2022}, solo el 54.7\% de los hogares tiene acceso a internet, siendo las zonas rurales las más desfavorecidas. Investigaciones como la de \textcite{vasquez2021} destacan que los estudiantes universitarios enfrentan dificultades para acceder a recursos digitales, especialmente durante la pandemia de COVID-19, lo que impactó negativamente en su rendimiento académico.

\textbf{Locales (Huánuco):} En la región de Huánuco, la brecha digital es particularmente pronunciada debido a la falta de infraestructura tecnológica y recursos educativos. Según un informe de la \textcite{dreh2023}, menos del 40\% de las instituciones educativas cuenta con acceso a internet de calidad.

\subsection{Bases Teóricas}

\textbf{Brecha digital:} La brecha digital es entendida como la desigualdad en el acceso, uso y aprovechamiento de las TIC \parencite{unesco2019}. \textcite{vandijk2020} propone que se manifiesta en tres niveles: acceso físico, habilidades para el uso y beneficios sociales o económicos.

\textbf{Educación y TIC:} La integración de las TIC en la educación ha transformado los procesos de enseñanza. Según \textcite{coll2021}, las desigualdades en acceso perpetúan diferencias en la calidad educativa.

\textbf{Competencias digitales:} Según el marco europeo DigComp \parencite{ferrari2013}, estas incluyen en la alfabetización informacional, comunicación, creación de contenido, seguridad y resolución de problemas.

\subsection{Definición Conceptual}

Para efectos de este estudio, se entiende por \textit{brecha digital} la desigualdad existente en el acceso, uso y desarrollo de habilidades relacionadas con las TIC entre los estudiantes de la Facultad de Economía de la UNHEVAL. Además, la \textit{inclusión digital} se define como el proceso de garantizar acceso y capacidad para utilizar tecnologías digitales \parencite{unesco2019}.


\section{Metodología}

La presente investigación adopta un enfoque cuantitativo, ya que busca medir y analizar de manera objetiva las desigualdades digitales en los estudiantes de la Facultad de Economía de la Unheval. Este enfoque permite la recopilación de datos numéricos para describir y comprender la magnitud y las características de la brecha digital en este contexto educativo específico.  

El diseño del estudio es no experimental, dado que no se manipulan las variables, sino que se observa y analiza la situación tal como ocurre en su entorno natural. En este sentido, la investigación se limita a describir la brecha digital sin interferir en las condiciones preexistentes.  

El nivel de investigación es descriptivo, ya que su principal objetivo es caracterizar la brecha digital en términos de acceso, uso y habilidades digitales entre los estudiantes. Este nivel de análisis permite identificar patrones y tendencias que ayuden a comprender mejor las desigualdades digitales en esta población.  

La técnica empleada para la recolección de datos es la encuesta, una herramienta idónea para obtener información directa de los estudiantes sobre su acceso a dispositivos tecnológicos, conectividad, frecuencia de uso de las TIC y competencias digitales.  

El instrumento utilizado será un cuestionario, diseñado específicamente para medir los aspectos clave relacionados con la brecha digital: acceso a dispositivos, calidad de la conectividad, frecuencia de uso de recursos digitales y nivel de habilidades en el manejo de herramientas tecnológicas. Este cuestionario será validado previamente mediante pruebas piloto para garantizar su confiabilidad y pertinencia en el contexto de la investigación.  

La población de estudio está constituida por los 348 estudiantes matriculados en la Facultad de Economía de la UNHEVAL durante el ciclo académico 2024-2. De esta población, se seleccionó una muestra representativa de 72 estudiantes, utilizando un método de muestreo probabilístico. Este tamaño muestral garantiza la representatividad de los resultados y permite realizar inferencias sobre la totalidad de la población estudiada.  

En resumen, la metodología diseñada para esta investigación asegura un enfoque sistemático y riguroso para analizar la brecha digital en los estudiantes de la Facultad de Economía de la UNHEVAL. Los datos obtenidos serán fundamentales para caracterizar las desigualdades digitales y sentar las bases para la formulación de estrategias que promuevan la inclusión y equidad digital en la universidad.

\subsection{Procedimiento de Recolección de Datos}

La recolección de datos se llevará a cabo mediante la distribución del cuestionario en línea a través de la plataforma educativa de la universidad y por correo electrónico institucional. Se implementarán recordatorios periódicos para maximizar la tasa de respuesta. El período de recolección de datos se extenderá por 2 semanas, permitiendo a los estudiantes completar el cuestionario en el momento que les resulte más conveniente.

\subsection{Análisis de Datos}

El análisis estadístico de los datos se realizará utilizando Python, aprovechando las bibliotecas especializadas para análisis de datos y visualización, como pandas, numpy, scipy, y matplotlib. La elección de Python se basa en su versatilidad, capacidad de manejo de grandes conjuntos de datos y la posibilidad de implementar análisis estadísticos avanzados.

El proceso de análisis de datos incluirá las siguientes etapas:

\begin{enumerate}
    \item Preparación y limpieza de datos: Utilizando pandas para la manipulación de datos y detección de valores atípicos o faltantes.
    \item Análisis descriptivo: Cálculo de estadísticas descriptivas (medias, medianas, desviaciones estándar) y generación de visualizaciones para comprender la distribución de las variables.
    \item Análisis inferencial: Aplicación de pruebas estadísticas como correlaciones, pruebas t, ANOVA, y regresiones múltiples para explorar relaciones entre variables.
    \item Análisis de conglomerados: Utilización de técnicas de machine learning no supervisado para identificar patrones en el uso de TIC entre los estudiantes.
    \item Visualización de resultados: Creación de gráficos y visualizaciones interactivas para representar los hallazgos de manera clara y efectiva.
\end{enumerate}

Se utilizará el nivel de significancia estándar de $\alpha = 0.05$ para todas las pruebas estadísticas. Los resultados se interpretarán en el contexto de la literatura existente y las teorías presentadas en el marco teórico.

\subsection{Consideraciones Éticas}

La investigación se llevará a cabo siguiendo estrictos principios éticos. Se obtendrá el consentimiento informado de todos los participantes antes de su inclusión en el estudio. La participación será voluntaria y anónima, y los datos recolectados se utilizarán exclusivamente para fines de investigación. 

Además, se tomarán medidas para garantizar la confidencialidad y seguridad de los datos recolectados, incluyendo el almacenamiento en servidores seguros y la eliminación de cualquier información que pueda identificar individualmente a los participantes.

Esta metodología proporciona un enfoque sistemático y riguroso para abordar los objetivos de la investigación, permitiendo una exploración profunda del uso y acceso a las TIC entre los estudiantes de economía, así como su relación con las competencias digitales y el rendimiento académico.


\newpage

\end{document}