\documentclass[12pt]{article}
\usepackage[utf8]{inputenc}
\usepackage{amsmath}
\usepackage[margin=2.54cm]{geometry}
\usepackage{graphicx}

\title{Examen de la Unidad 2}
\author{Universidad Nacional Hermilio Valdizán \\ Facultad de Economía \\ Estadística y Probabilidad}
\date{}

\begin{document}

\maketitle

\begin{center}
\textbf{Profesor: Jeel Cueva} \\ 
\textbf{Grupo 1}
\end{center}

\section*{Problemas}

\textbf{1. Demostrar la derivación de la fórmula de la Mediana (Me) para datos clasificados.} (5 puntos)

\vspace{1cm}

\textbf{2. Dada la siguiente tabla de frecuencia no común:}

\begin{center}
\begin{tabular}{|c|c|}
\hline
\textbf{Intervalos} & \textbf{Frecuencia absoluta} \\
\hline
20-40 & 10 \\
40-50 & 25 \\
50-80 & 46 \\
80-90 & 9 \\
90-94 & 10 \\
\hline
\textbf{Total} & 100 \\
\hline
\end{tabular}
\end{center}

\textbf{a) Determinar la media y la mediana de esta distribución.} (2 puntos)

\textbf{b) Hallar el procentaje de los datos contenidos en el intervalo.} (3 punto)

$I = \{ x \in R / |x - X|< \bar{X}/3  \}$. Donde, $\bar{X}$ es la media, $X$ es la mediana, ambas para datos clasificados.

\vspace{1cm}

\textbf{3. La siguiente tabla representa el número de trabajadores encuestados en 60 empresas en el mes de octubre en el departamento de Huánuco en 2024:}

\begin{center}
\begin{tabular}{|c|c|c|c|c|c|c|c|c|c|}
\hline
282 & 252 & 254 & 305 & 295 & 301 & 293 & 294 & 295 & 302\\
270 & 270 & 275 & 292 & 280 & 297 & 230 & 262 & 253 & 263\\
251 & 255 & 266 & 274 & 268 & 258 & 282 & 284 & 268 & 287\\
303 & 287 & 295 & 269 & 251 & 266 & 270 & 276 & 274 & 250\\
265 & 260 & 297 & 271 & 270 & 280 & 265 & 271 & 275 & 260\\
294 & 263 & 253 & 254 & 289 & 262 & 255 & 261 & 282 & 261\\
\hline
\end{tabular}
\end{center}

\textbf{a) Elaborar la tabla de frecuencias usando la regla de Sturges.} (2 puntos)

\textbf{b) Interpretar: $f_5$, $h_2$.} (2 puntos)

\textbf{c) Elaborar el diagrama escalonado y la ojiva respectiva.} (2 puntos)

\textbf{d) Calcular la media aritmética ($\bar{x}$) y la mediana (Me) para datos agrupados y no agrupados.} (2 puntos)

\textbf{e) Calcular la moda (Mo) para datos agrupados y no agrupados.} (2 puntos)

\end{document}
