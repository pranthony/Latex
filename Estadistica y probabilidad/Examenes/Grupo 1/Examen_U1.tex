\documentclass{article}
\usepackage[utf8]{inputenc}
\usepackage{amsmath}
\usepackage{graphicx}
\usepackage[a4paper, margin=2.54cm]{geometry}  % Configurar márgenes a 2.54 cm

\title{Examen - Unidad 1}
\author{Jeel Cueva}
\date{September 19, 2024}

\begin{document}

\maketitle

\section*{Universidad Nacional Hermilio Valdizán}
\subsection*{Facultad de Economía}

\subsubsection*{Estadística y Probabilidades (G1)}

\section*{Teoría}

\begin{enumerate}
    \item Demostrar las siguientes ecuaciones:
    \begin{enumerate}
        \item Demostrar o probar que la muestra ($n$) para una población infinita es:
\[
        n = \frac{z^2pq}{e^2}
\]
        Donde, $z$ es el valor crítico de la distribución normal estándar, $p$ representa la proporción estimada de la población que tiene la característica de interés, $q$ significa que es la proporción de la población que no tiene la característica de interés y $e$ es la cantidad de error tolerable en la estimación de la proporción. (5 puntos)
        
        \item Demostrar que:
\[
        k = 1 + 3.322 \log_{10}(n)
\]
        Donde, $k$ es el número de intervalos o clases y $n$ es el número de observaciones. (5 puntos)
    \end{enumerate}
\end{enumerate}

\section*{Práctica}

\begin{enumerate}
    \item Responder los siguientes casos:
    \begin{enumerate}
        \item La siguiente tabla representa el número de trabajadores que fueron encuestados a 60 empresas en el mes de agosto en 2024 en la región de Huánuco.
        
        \begin{verbatim}
        282 252 254 305 295 301 293 294 295 302
        270 270 275 292 280 258 230 262 253 263
        251 255 266 274 268 297 282 284 268 287
        303 287 295 269 251 266 270 276 274 250
        265 260 297 271 270 280 265 271 275 260
        294 263 253 254 289 262 255 261 282 261
        \end{verbatim}
        
        \begin{itemize}
            \item[a.1.] Construir una tabla de frecuencia, usando el algoritmo de Sturges. (2 puntos)
            \item[a.2.] Interpretar: $h_5$, $F_3$, $f_2$, \%5 (1 punto)
            \item[a.3.] Elaborar el histograma de frecuencia relativa, generar el polígono de frecuencia absoluta y generar las gráficas de torta para la frecuencia porcentual. (2 puntos)
        \end{itemize}
        
        \item Como control de la ética publicitaria, se requiere que el rendimiento en millas por galón de gasolina que los fabricantes de automóviles usan con fines publicitarios esté basado en un buen número de pruebas efectuadas en diversas condiciones. Al tomar una muestra de 50 automóviles, se registran las siguientes observaciones en millas por galón:
        
        \begin{verbatim}
        35.6 27.9 29.3 31.8 22.5 34.2 34.2 32.7 26.5 26.4
        31.0 31.6 28.0 33.7 32.0 28.5 27.5 29.8 31.2 28.7
        30.0 28.7 33.2 30.5 27.9 31.2 29.5 28.7 23.0 30.1
        30.5 31.3 24.9 26.8 29.9 28.7 30.4 31.3 32.7 30.3
        33.5 30.5 30.6 35.1 28.6 30.1 30.3 29.6 31.4 32.4
        \end{verbatim}
        
        \begin{itemize}
            \item[b.1.] Construya un Histograma de frecuencias relativas usando 5 intervalos de clase de la misma longitud.
            \item[b.2.] Los fabricantes afirman que su automóvil está diseñado para rendir al menos 30 millas por galón. ¿Qué porcentaje de autos en la muestra tiene ese rendimiento?
            \item[b.3.] Representar gráficamente las distribuciones de las frecuencias obtenidas en (b.1).
        \end{itemize}
    \end{enumerate}
\end{enumerate}

\section*{Nota}

La duración del examen es de 2 horas. Se recomienda el uso de una calculadora científica, lapicero y hojas tamaño oficio.

Les deseo mucho éxito en su examen.

\textbf{Atentamente:} J Cueva

\end{document}
