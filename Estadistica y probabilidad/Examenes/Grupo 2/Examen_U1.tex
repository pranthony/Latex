\documentclass{article}
\usepackage[utf8]{inputenc}
\usepackage{amsmath}
\usepackage{graphicx}
\usepackage[a4paper, margin=2.54cm]{geometry}  % Configurar márgenes a 2.54 cm

\title{Examen - Unidad 1}
\author{Jeel Cueva}
\date{September 19, 2024}

\begin{document}

\maketitle

\section*{Universidad Nacional Hermilio Valdizán}
\subsection*{Facultad de Economía}

\subsubsection*{Estadística y Probabilidades (G2)}

\section*{Teoría}

\begin{enumerate}
    \item Demostrar las siguientes ecuaciones:
    \begin{enumerate}
        \item Demostrar o probar que la muestra ($n$) para una población finita es:
\[
        n = \frac{z^2pqN}{e^2(N-1)+z^2pq}
\]
        Donde, $z$ es el valor crítico de la distribución normal estándar, $p$ representa la proporción estimada de la población que tiene la característica de interés, $q$ significa que es la proporción de la población que no tiene la característica de interés, $e$ es la cantidad de error tolerable en la estimación de la proporción y N es el tamaño de población. (5 puntos)
        
        \item Demostrar que:
\[
        k = 1 + 3.322 \log_{10}(n)
\]
        Donde, $k$ es el número de intervalos o clases y $n$ es el número de observaciones. (5 puntos)
    \end{enumerate}
\end{enumerate}

\section*{Práctica}

\begin{enumerate}
    \item Responder los siguientes casos:
    \begin{enumerate}
        \item Considere los datos de las medidas de altura de los estiantes de Economía de la Unheval en cm.
        
        \begin{verbatim}
    151, 152, 154, 158, 159, 170, 152, 158, 189, 190,
    161, 162, 163, 163, 165, 160, 162, 163, 188, 182,
    166, 166, 166, 167, 167, 178, 173, 167, 187, 178,
    168, 168, 168, 168, 168, 175, 184, 168, 170, 176,
    169, 169, 169, 169, 170, 172, 188, 169, 156, 173,
    170, 170, 171, 171, 187, 168, 150, 174, 177, 170,
    173, 173, 174, 174, 188, 167, 156, 171, 181, 169,
    176, 176, 176, 177, 190, 166, 166, 177, 185, 168,
    179, 179, 180, 180, 190, 186, 162, 181, 152, 166,
    182, 182, 183, 185, 187, 150, 158, 185, 159, 161
        \end{verbatim}
        
        \begin{itemize}
            \item[a.1.] Construir una tabla de frecuencia, usando el algoritmo de Sturges. (2 puntos)
            \item[a.2.] Interpretar: $h_5$, $F_3$, $f_2$, \%5 (1 punto)
            \item[a.3.] Elaborar el histograma de frecuencia relativa, generar el polígono de frecuencia absoluta y generar las gráficas de torta para la frecuencia porcentual. (2 puntos)
        \end{itemize}
        
        \item Las velocidades de los rayos X para tratamiento médico en el hospital Hermilio Valdizán de Huánuco fueron registrados en millisegundos (1/1000 de un segundo) y fueron.
        
        \begin{verbatim}
    0.3, 0.9, 1.1, 1.7, 1.5, 0.8, 0.7, 1.1,
    0.8, 1.0, 1.3, 0.2, 1.6, 0.1, 0.5, 0.7,
    1.2, 1.5, 0.8, 0.9, 0.7, 0.5, 1.1, 1.5,
    0.1, 1.4, 0.7, 0.8, 0.6, 1.3, 1.2, 1.4,
    1.8, 0.7, 0.9, 1.0, 0.3, 1.2, 1.8, 1.0
        \end{verbatim}
        
        \begin{itemize}
            \item[b.1.] Construya una distribución de frecuencia usando intervalos de tamaño 0.25 millisegundos.
            \item[b.2.] Construya un histograma y un poligono de frecuencia a partir de los datos.
            \item[b.3.] Construya una ojiva de frecuencia relativa (mayor que) a partir de los datos.
            \item[b.4.] Interpretar: $h_6$, $F_5$, $f_3$, \%1 (1 punto) 
        \end{itemize}
    \end{enumerate}
\end{enumerate}

\section*{Nota}

La duración del examen es de 2 horas. Se recomienda el uso de una calculadora científica, lapicero y hojas tamaño oficio.

Les deseo mucho éxito en su examen.

\textbf{Atentamente:} J Cueva

\end{document}
