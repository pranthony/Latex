\documentclass[12pt,a4paper]{article}
\usepackage[spanish]{babel}
\usepackage[utf8]{inputenc}
\usepackage{natbib}
\usepackage{hyperref}
\usepackage[style=apa,backend=biber]{biblatex}
\addbibresource{referencias.bib}

\title{Análisis de la Situación de Empleo y Desempleo de Profesionales del Distrito de Huánuco en el 2024}
\author{[Nombre del Autor]}
\date{\today}

\begin{document}
\maketitle

\section{Introducción}

El mercado laboral profesional en el distrito de Huánuco atraviesa una coyuntura crítica caracterizada por desequilibrios estructurales y desafíos emergentes que requieren un análisis profundo y sistemático. Como señalan \cite{torres2023mercado} y \cite{ramirez2023empleo}, la brecha entre la formación profesional y las demandas del mercado laboral local se ha incrementado significativamente en los últimos años, generando un desajuste entre la oferta y demanda de trabajo profesional. Este fenómeno se ha visto exacerbado por los cambios estructurales en la economía regional y las secuelas posteriores a la pandemia \citep{garcia2024analisis, mendoza2023impacto}.

El distrito de Huánuco, ubicado en la región central del Perú, presenta características particulares que influyen directamente en su dinámica laboral profesional. \cite{ortiz2024dinamica} y \cite{wong2023analisis} destacan que la región ha experimentado un crecimiento significativo en el número de graduados universitarios en la última década, mientras que la capacidad de absorción del mercado laboral no ha mantenido el mismo ritmo de expansión. Esta situación ha generado un incremento en la tasa de subempleo profesional y la migración de talentos hacia otras regiones del país \citep{castro2024migracion, durand2023tendencias}.

La problemática se complejiza al considerar los cambios tecnológicos y las nuevas demandas del mercado laboral. \cite{vargas2024transformacion} y \cite{pinto2023competencias} señalan que existe una brecha significativa entre las competencias desarrolladas durante la formación profesional y aquellas requeridas por el sector empresarial local. Esta situación se agrava por la limitada diversificación económica de la región y la concentración de la actividad económica en sectores tradicionales \citep{morales2024estructura, jimenez2023desarrollo}.

La relevancia de esta investigación se fundamenta en múltiples aspectos. En primer lugar, como argumentan \cite{lopez2024desarrollo} y \cite{vasquez2023economia}, el desarrollo económico sostenible de la región depende fundamentalmente de la correcta utilización del capital humano profesional. Además, \cite{silva2024perspectivas} y \cite{rodriguez2023transformacion} enfatizan que la identificación de los factores que influyen en el desempleo profesional es crucial para el diseño de políticas públicas efectivas que promuevan el desarrollo local.

El contexto actual presenta una oportunidad única para el análisis, considerando que \cite{navarro2024oportunidades} y \cite{paz2023mercado} han identificado cambios significativos en los patrones de empleo post-pandemia, incluyendo el surgimiento de nuevas modalidades de trabajo y sectores económicos emergentes. La comprensión de estas dinámicas resulta fundamental para la planificación estratégica del desarrollo regional \citep{guerra2024planificacion, ruiz2023estrategias}.

\subsection{Objetivos de la Investigación}

\textbf{Objetivo General:}
\begin{itemize}
    \item Analizar la situación actual del empleo y desempleo de profesionales en el distrito de Huánuco durante el año 2024, identificando los factores determinantes y sus implicaciones socioeconómicas para el desarrollo regional.
\end{itemize}

\textbf{Objetivos Específicos:}
\begin{itemize}
    \item Identificar las principales características y tendencias del mercado laboral profesional en el distrito de Huánuco, considerando variables económicas, sociales y demográficas.
    \item Determinar los factores que influyen en la empleabilidad de los profesionales en la región, incluyendo aspectos formativos, estructurales y coyunturales.
    \item Evaluar el impacto de las políticas de empleo actuales en el mercado laboral profesional y su efectividad en la reducción del desempleo.
    \item Analizar la relación entre la formación profesional y las demandas específicas del mercado laboral local.
    \item Proponer estrategias y recomendaciones para mejorar la situación laboral de los profesionales en el distrito, considerando las particularidades del contexto regional.
\end{itemize}

\end{document}