\documentclass[12pt]{article}
\usepackage[utf8]{inputenc}
\usepackage{amsmath}
\usepackage[margin=2.54cm]{geometry}
\usepackage{graphicx}

\title{Examen de la Unidad 2}
\author{Universidad Nacional Hermilio Valdizán \\ Facultad de Economía \\ Estadística y Probabilidad}
\date{}

\begin{document}

\maketitle

\begin{center}
\textbf{Profesor: Jeel Cueva} \\ 
\textbf{Grupo 2}
\end{center}

\section*{Problemas}

\textbf{1. Demostrar la derivación de la fórmula de la Moda (Mo) para datos clasificados.} \\
(5 puntos)

\vspace{1cm}

\textbf{2. La compañía Petro - Perú maneja una pequeña refinería en Ica que vende gasolina al por mayor a mayoristas independientes. Las ventas de la semana pasada fueron las siguientes:}

\begin{center}
\begin{tabular}{|c|c|}
\hline
\textbf{Galones de gasolina (en miles)} & \textbf{Número de operaciones} \\
\hline
[0-10) & 10 \\
10-20 & 20 \\
20-30 & 30 \\
30-40 & 25 \\
40-50 & 15 \\
50-60 & 10 \\
60-70 & 5 \\
70-80 & 5 \\
\hline
\textbf{Total} & 120 \\
\hline
\end{tabular}
\end{center}

\textbf{a) A partir de esta distribución de frecuencias, calcule el número total de galones vendidos la semana pasada.} (2 puntos)

\textbf{b) Determina la media de los galones vendidos en cada operación.} (1 punto)

\textbf{c) ¿La moda se encuentra por debajo o por arriba de los 25 000 galones? ¿Cómo lo sabe?} (1 punto)

\textbf{d) Calcule la mediana de las ventas.} (1 punto)

\vspace{1cm}

\textbf{3. La siguiente tabla representa el número de trabajadores encuestados en 56 empresas en el mes de octubre en el departamento de Huánuco en 2024:}

\begin{center}
\begin{tabular}{|c|c|c|c|c|c|c|c|}
\hline
73 & 95 & 61 & 46 & 70 & 55 & 87 & 65 \\
75 & 48 & 69 & 75 & 75 & 39 & 63 & 82 \\
58 & 43 & 38 & 64 & 69 & 79 & 47 & 63 \\
63 & 81 & 59 & 77 & 84 & 34 & 75 & 93 \\
67 & 89 & 66 & 52 & 59 & 36 & 62 & 43 \\
75 & 52 & 59 & 87 & 74 & 30 & 95 & 38 \\
50 & 72 & 44 & 53 & 68 & 72 & 82 & 63 \\
\hline
\end{tabular}
\end{center}

\textbf{a) Elaborar la tabla de frecuencias usando la regla de Sturges.} (2 puntos)

\textbf{b) Interpretar: $f_5$, $h_2$.} (2 puntos)

\textbf{c) Elaborar el diagrama escalonado y la ojiva respectiva.} (2 puntos)

\textbf{d) Calcular la media aritmética ($\bar{x}$) y la mediana (Me) para datos agrupados y no agrupados.} (2 puntos)

\textbf{e) Calcular la moda (Mo) para datos agrupados y no agrupados.} (2 puntos)

\end{document}
