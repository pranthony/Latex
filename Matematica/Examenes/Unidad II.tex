\documentclass[12pt,a4paper]{article}
\usepackage[utf8]{inputenc}
\usepackage[T1]{fontenc}
\usepackage{amsmath}
\usepackage{amsfonts}
\usepackage{amssymb}
\usepackage[left=2.54cm,right=2.54cm,top=2.54cm,bottom=2.54cm]{geometry}
\usepackage{setspace}
\usepackage{titlesec}
\usepackage{fancyhdr}
\usepackage{natbib}
\usepackage{graphicx}
\usepackage{multicol}

% Configuración de APA 7ma edición
\onehalfspacing
\setlength{\parindent}{0.5in}
\titleformat*{\section}{\normalfont\bfseries}
\titleformat*{\subsection}{\normalfont\bfseries}
\titleformat*{\subsubsection}{\normalfont\bfseries}
\fancyhf{}
\rhead{\thepage}
\begin{document}
\setlength{\columnsep}{1cm} 


\begin{center}
\textbf{MATEMÁTICA II}\\
\textbf{EVALUACIÓN DE COMPETENCIAS}\\
\textbf{SEGUNDA UNIDAD}
\end{center}

\noindent
\textbf{Apellidos y Nombres:} Palomino Ricaldi Antony

\begin{enumerate}
    \item En una empresa textil de Huánuco dedicada a la confección de polos. Se venden 200 polos cuando el precio es de S/ 40 y 160 cuando el precio es de S/ 20.
    \begin{enumerate}
        \item Determina la ecuación de oferta.
        \begin{multicols}{2}
        \noindent\textit{Nota:} \(m = \frac{y_2 - y_1}{x_2 - x_1}\)
        
        \begin{center}
        \begin{tabular}{|c|c|}
        \hline
        Q & P \\
        \hline
        200 & 40 \\
        160 & 20 \\
        \hline
        \end{tabular}
        \end{center}
        
        La ecuación de oferta tiene la forma:
        \[
        y = mx + b
        \]
        Donde \( m \):
        \[
        m = \frac{40 - 20}{200 - 160} = \frac{20}{40} = \frac{1}{2}
        \]
        Y \( b \):
        \[
        b = 40 - 200 \cdot \frac{1}{2} = 40 - 100 = -60
        \]
        Entonces la ecuación de oferta es:
        \[
        y = \frac{1}{2}x - 60
        \]
        \item Cuando el precio es de S/ 30, ¿cuántos polos pueden vender?
        \[
        30 = \frac{1}{2}x - 60
        \]
        \[
        90 = \frac{1}{2}x
        \]
        \[
        x = 180
        \]
        
        \item Si se determina vender 245 polos, ¿cuál sería el precio?
        \[
        y = \frac{1}{2}(245) - 60
        \]
        \[
        y = 122.5 - 60 = 62.5
        \]
        \end{multicols}
        
        \item Si el precio de equilibrio del polo fuera 40 soles, ¿qué pasaría si el precio disminuye a 30 soles?
        
        Si el precio disminuye a 30 soles, la cantidad ofertada se reduce a 180 polos, lo que ocasionaría un déficit en el mercado de polos.
    \end{enumerate}
    
    \item Una empresa produce y vende bicicletas en un mercado. La empresa quiere encontrar el punto de equilibrio entre la oferta y la demanda del bien dadas las siguientes ecuaciones:
    \[
    P_1 = 2q + 50
    \]
    \[
    P_2 = -5q + 200
    \]
    Por método de igualación:
    \[
    2q + 50 = -5q + 200
    \]
    \[
    7q = 150
    \]
    \[
    q = 21
    \]
    
    \noindent\textit{Nota:} Para cantidad redondear al entero próximo, debido a que no puedes producir bicicletas en fracciones (dato discreto).
    
    Reemplazando para calcular el precio:
    \[
    P_1 = 2(21) + 50 = 92
    \]

    \item 
    Halla las pendientes y determina si es perpendicular o paralelo:
    \begin{multicols}{2}
        
    \[
    2x + 8y - 32 = 0 \tag{1}
    \]
    \[
    -8x + 2y - 16 = 0 \tag{2}
    \]
    
    Despejando (1):
    \[
    8y = 32 - 2x
    \]
    \[
    y = \frac{32 - 2x}{8}
    \]
    \[
    y = 4 - \frac{x}{4}
    \]
    
    Despejando (2):
    \[
    2y = 16 + 8x
    \]
    \[
    y = \frac{16 + 8x}{2} = 4x + 8
    \]

    Calcular si son perpendiculares, para lo cual debe cumplirse que el producto de las pendientes es igual a -1:

    \[ m_1 \times m_2 = -1 \]
    \[ -\frac{1}{4} \times 4 = -1 \]
    
    \end{multicols}
    
    Por lo tanto, no es \textbf{paralelo} porque las pendientes de (1) y (2) son diferentes y son \textbf{perpediculares} porque el producto de sus pendientes es igual a -1.
    
    \item Dado el sistema de ecuaciones:
    \begin{multicols}{2}
        
    \[
    x + 2y + z = 7 \tag{1}
    \]
    \[
    3x + y + z = 5 \tag{2}
    \]
    \[
    2x + 3y - z = 3 \tag{3}
    \]
    
    Hallar los valores de \(x\), \(y\) y \(z\) por el método de reducción.
    
    Restando (1) - (2):
    \[
    x + 2y + z = 7 \tag{1}
    \]
    \[
    3x + y + z = 5 \tag{2}
    \]
    \[
    -2x + y = 2 \tag{4}
    \]
    
    Sumando (2) a (3):
    \[
    3x + y + z = 5 \tag{2}
    \]
    \[
    2x + 3y - z = 3 \tag{3}
    \]
    \[
    5x + 4y = 8 \tag{5}
    \]
    
    Multiplicando (4) por 4:
    \[
    -8x + 4y = 8 \tag{6}
    \]
    
    Restando (5) - (6):
    \[
    5x + 4y = 8 \tag{5}
    \]
    \[
    -8x + 4y = 8 \tag{6}
    \]
    \[
    13x = 0 \tag{7}
    \]
    
    Por lo tanto, \(x = 0\).
    
    Reemplazando \(x\) en (5):
    \[
    5(0) + 4y = 8 \tag{5}
    \]
    \[
    y = 2
    \]
    
    Reemplazando \(x\) y \(y\) en (1):
    \[
    0 + 2(2) + z = 7 \tag{1}
    \]
    \[
    z = 3
    \]
    
    Comprobando valores de \(x\), \(y\) y \(z\) en (2):
    \[
    3(0) + 2 + 3 = 5 \tag{2}
    \]
    
    Por lo tanto, el conjunto solución es:
    \[
    CS = (0, 2, 3)
    \]
    \end{multicols}
    
    \item Una refinería recibe de Arabia, Venezuela y México petróleo para destilar crudo y obtener tres productos destilados: keroseno, gasolina y gas-oil en las siguientes proporciones:
    
    \begin{center}
    \begin{tabular}{|c|c|c|c|}
    \hline
    & GASOLINA & KEROSENE & GAS-OIL \\
    \hline
    ARABIA & 0.4 & 0.4 & 0.2 \\
    MÉXICO & 0.2 & 0.6 & 0.2 \\
    VENEZUELA & 0.4 & 0.1 & 0.5 \\
    \hline
    \end{tabular}
    \end{center}
    
    La empresa de refinamiento ha firmado un contrato con una empresa de distribución para suministrar 10000 barriles de gasolina, 5000 barriles de gas-oil y 2000 barriles de keroseno.
    
    \begin{enumerate}
        \item Plantear un modelo matemático que permita obtener el número de barriles que se deben destilar de cada crudo si se desea cumplir el contrato, sin que sobre ningún barril.
        
        \noindent\textit{Nota:} Arabia (\(x\)), México (\(y\)) y Venezuela (\(z\)).
        
        Los modelos para cada tipo de crudo serían:
        
        Gasolina:
        \[
        0.4x + 0.2y + 0.4z = 10000
        \]
        Keroseno:
        \[
        0.4x + 0.6y + 0.1z = 2000
        \]
        Gas-oil:
        \[
        0.2x + 0.2y + 0.5z = 5000
        \]
        
        \item Encontrar el número de barriles a destilar de cada tipo de crudo por el método de determinantes.
        \begin{multicols}{2}
            
        \[
        \Delta_s = \begin{vmatrix} 0.4 & 0.2 & 0.4 \\ 0.4 & 0.6 & 0.1 \\ 0.2 & 0.2 & 0.5 \end{vmatrix} = 0.06
        \]
        \[
        \Delta_x = \begin{vmatrix} 10000 & 0.2 & 0.4 \\ 2000 & 0.6 & 0.1 \\ 5000 & 0.2 & 0.5 \end{vmatrix} = 1660
        \]
        \[
        \Delta_y = \begin{vmatrix} 0.4 & 10000 & 0.4 \\ 0.4 & 2000 & 0.1 \\ 0.2 & 5000 & 0.5 \end{vmatrix} = -960
        \]
        \[
        \Delta_z = \begin{vmatrix} 0.4 & 0.2 & 10000 \\ 0.4 & 0.6 & 2000 \\ 0.2 & 0.2 & 5000 \end{vmatrix} = 320
        \]
        
        Entonces:
        \[
        x = \frac{1660}{0.06} = 27666.\overline{6}
        \]
        \[
        y = \frac{-960}{0.06} = -16000
        \]
        \[
        z = \frac{320}{0.06} = 5333.\overline{3}
        \]
        
        Comprobando los resultados:
        \[
        0.4(27666.\overline{6}) + 0.2(-16000) 
        \]
        \[
        + 0.4(5333.\overline{3}) = 10000
        \]
        

        Por lo tanto, el conjunto solución es:
        \[
        CS = (27666.\overline{6}, -16000, 5333.\overline{3})
        \]
        \end{multicols}
        
        Interpretación de los resultados:
        \begin{itemize}
            \item \(x \approx 27667\) barriles deben destilarse de Arabia.
            \item \(y \approx -16000\) barriles deben destilarse de México (este valor negativo es un problema, ya que no tiene sentido físico).
            \item \(z \approx 5333\) barriles deben destilarse de Venezuela.
        \end{itemize}
        
        El valor negativo para \(y\) sugiere que el sistema planteado tiene algún problema. En particular, puede indicar que el conjunto de proporciones o las cantidades exigidas no permiten una solución factible.
    \end{enumerate}
\end{enumerate}

\end{document}
